\documentclass[12pt]{report}
\usepackage[utf8]{inputenc}
%\usepackage{natbib}
\usepackage{graphicx}
\usepackage[dvipsnames]{xcolor}
\usepackage[spanish]{babel} 
\usepackage{actuarialangle}
\usepackage{actuarialsymbol}
\usepackage[backend=bibtex]{biblatex}

\addbibresource{includes/bibliografia.bib}

\renewcommand{\theequation}{\thesubsection.\arabic{equation}}
\newcounter{neq}

\usepackage{Sweave}
\begin{document}

\begin{titlepage}
	\centering
	{\scshape\LARGE\bfseries UNIVERSIDAD NACIONAL AGRARIA LA MOLINA \par}
	\vspace{0.60cm}	
	{\scshape\large\bfseries ESCUELA DE POSGRADO \par}
	\vspace{0.60cm}	
	{\scshape\large\bfseries MAESTR\'IA EN ESTAD\'ISTICA APLICADA  \par}
	\vspace{0.60cm}	
	
	\includegraphics[width=0.30\textwidth]{img/323x386_ESCUDOCOLOR}\par\vspace{1cm}
	{\scshape\large 2019-1 : CURSO ESTAD\'ISTICA ACTUARIAL\par}
	\vspace{0.60cm}
	{\large\bfseries TRABAJO FINAL "M\'ETODOS DE INTERPOLACI\'ON PARA EDADES FRACCIONALES" \par}
	\vspace{0.60cm}

	\vfill
	Presentado por \par
	{\large\itshape { Jaime G\'omez Mar\'in \\ Roberto Le\'on Leiva \\ Arturo Zu\~niga Blanco}\par}
	\vspace{0.40cm}
	\vfill
	Docente \par
	Msc.~Jes\'us Eduardo \textsc{Gamboa Unsihuay}
	
  \vspace{0.40cm}
	\vfill
% Bottom of the page
	{\large \today\par}
\end{titlepage}


\tableofcontents  

\Sconcordance{concordance:Trabajo-Final.tex:Trabajo-Final.Rnw:%
1 15 1 1 0 228 1}


\chapter{Presentaci\'on}

\chapter{Introducci\'on}

\chapter{Marco Te\'orico}

\section{Supuestos de las Edades Fraccionarias}
\setcounter{equation}{0}

Las tablas de vida generalmente muestra el n\'umero de personas vivas a edades exactas. Si necesitamos informaci\'on de las tablas de vida sobre una fracci\'on de un a\~no, debemos hacer suposiciones con respecto a la tabla. Existen 3 supuestos:

\begin{itemize}
\item Primer supuesto : Distribuci\'on Uniforme de Muertes (UDD)
\item Segundo supuesto : Fuerza de Mortalidad constante
\item Tercer supuesto : Supuesto Balducci
\end{itemize}


\subsection{Distribuci\'on Uniforme de Muertes (UDD)}

\paragraph{Primer supuesto : \textit{Distribuci\'on Uniforme de las muertes.}}
Asume que la distribuci\'on uniforme de muertes sigue una regresi\'on lineal.

%Un supuesto común es el de una Distribución Uniforme de Muertes (UDD) en cada año de edad. Bajo este supuesto, {\ displaystyle \, l_ {x + t}} \, l _ {{x + t}} es una interpolación lineal entre {\ displaystyle \, l_ {x}} \, l_x y {\ displaystyle \, l_ {x + 1}} \, l _ {{x + 1}}. es decir





\subsection{Fuerza de Mortalidad constante}

Partamos de la definici\'on de fuerza de mortalidad: \cite{TheForceOfMortality2013} "\textit{En estad\'istica actuarial representa la tasa de mortalidad instantanea a una cierta edad dentro una base anualizada}"

\begin{equation}
u_x = \lim_{\Delta x \rightarrow 0 } \frac{P(x < X < x + \Delta x | X > x) }{\Delta x}
\addtocounter{neq}{1}
\end{equation}

recordando que:

\begin{equation}
P(x < X < x + \Delta x | X > x) = \frac{F(x + \Delta x ) - F(x)}{1 - F(x)}
\addtocounter{neq}{1}
\end{equation}

reemplazando:

\begin{equation*}
u_x 
= \lim_{\Delta x \rightarrow 0 } \frac{F(x + \Delta x ) - F(x) }{\Delta x *(1 - F(x)) } 
\addtocounter{neq}{1}
\end{equation*}

\begin{equation*}
u_x
= \frac{1}{(1 - F(x))}*\lim_{\Delta x \rightarrow 0 } \frac{F(x + \Delta x ) - F(x) }{\Delta x}
\addtocounter{neq}{1}
\end{equation*}

\begin{equation*}
u_x
= \frac{1}{(1 - F(x))}*F'(x)
\addtocounter{neq}{1}
\end{equation*}

\begin{equation}
u_x
= \frac{F'(x)}{1 - F(x)}
\addtocounter{neq}{1}
\end{equation}

se sabe que:

\begin{equation*}
1 - F(x) = S(x)   
\addtocounter{neq}{1}
\end{equation*}

\begin{equation*}
F'(x) =  -S'(x)     
\addtocounter{neq}{1}
\end{equation*}


reemplazando:
\begin{equation}
u_x
= -\frac{S'(x)}{S(x)} 
= - \frac{d}{dx}ln(S(x))
\addtocounter{neq}{1}
\end{equation}


\hfill \break
Procedemos a integrar entre $x_0 < X < x_0 + t$ :

\begin{equation}
\label{eqn:ecFuerzaMortalidad}
\int_{x_0}^{x_0+t}u_xdx
=  ln( S(x_0))  - ln(S(x_0+t)) 
\addtocounter{neq}{1}
\end{equation}

\paragraph{Segundo supuesto : \textit{Fuerza de Mortalidad Constante}.}El  supuesto consiste en que la fuerza de mortalidad  $\mu$ se mantiene constante entre un rango de edades exactas; es decir para una edad exacta $x_o$, se tiene una variable X dentro de un rango de $x_0 \leq X < x_0 + t $ donde la fuerza de mortalidad no cambia. Lo anterior significa que la fuerza de mortalidad $\mu_{x_0}$ no depende del valor de t siempre y cuando t este entre los rangos de  $0 < t < 1$. A esta Fuerza de Mortalidad constante la vamos a denominar $\mu_x^*$. \\ 

En la ecuaci\'on  \ref{eqn:ecFuerzaMortalidad} aplicamos el segundo supuesto 
\begin{equation*}
\int_{x_0}^{x_0+1}\mu_x^*dx
=  ln( S(x_0))  - ln(S(x_0+1)) 
\addtocounter{neq}{1}
\end{equation*}

\begin{equation*}
\mu_x^*(x_0+1-x_0)
=  ln( S(x_0))  - ln(S(x_0+1)) 
\addtocounter{neq}{1}
\end{equation*}

\begin{equation}
\label{eqn:ecFuerzaMortalidadConstante}
\mu_x^*
=  ln( S(x_0))  - ln(S(x_0+1)) 
\addtocounter{neq}{1}
\end{equation}

En la ecuaci\'on  \ref{eqn:ecFuerzaMortalidad} aplicamos el segundo supuesto para un periodo de tiempo t 

\begin{equation*}
\int_{x_0}^{x_0+t}\mu_x^*dx
=  ln( S(x_0))  - ln(S(x_0+t)) 
\addtocounter{neq}{1}
\end{equation*}

\begin{equation*}
\mu_x^*t
=  ln( S(x_0))  - ln(S(x_0+t)) 
\addtocounter{neq}{1}
\end{equation*}

\begin{equation*}
ln(S(x_0+t))
=  ln( S(x_0))  - \mu_x^*t  
\addtocounter{neq}{1}
\end{equation*}

\begin{equation*}
ln(S(x_0+t))
=  ln(S(x_0))  - (ln(S(x_0))  - ln(S(x_0+1)))t  
\addtocounter{neq}{1}
\end{equation*}

\begin{equation}
ln(S(x_0+t))
=  (1-t)ln(S(x_0))  + t*ln(S(x_0+1))  
\addtocounter{neq}{1}
\end{equation}



\subsection{Supuesto Balducci}

\chapter{Aplicaci\'on}

\chapter{Conclusiones}


\printbibliography[
heading=bibintoc,
title={Bibliograf\'ia}
]

\printbibliography[heading=subbibintoc,type=article,title={Articles only}]


\end{document}
